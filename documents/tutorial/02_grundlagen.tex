\section{Grundlagen}
Einfache Textformatierung hilft dabei, wichtige Konzepte innerhalb eines Dokuments hervorzuheben und es lesbarer zu machen.

\subsection{Abstände}
''Unsichtbare'' Zeichen wie das Leerzeichen, Tabulatoren und das Zeilenende werden von \LaTeX\ einheitlich als Leerzeichen behandelt. \emph{Mehrere} Leerzeichen werden wie \emph{ein} Leerzeichen behandelt. Wenn man andere als die normalen Wort- und Zeilenabstände möchte, kann man dies also nicht durch die Eingabe von zusätzlichen Leerzeichen oder Leerzeilen erreichen, sondern nur mit entsprechenden \LaTeX\-Befehlen.\\
\emph{Eine} Leerzeile zwischen Textzeilen bedeutet das Ende eines Absatzes. \emph{Mehrere} Leerzeilen werden wie eine Leerzeile behandelt.

\vspace{3mm}
\noindent Sie haben die Möglichkeit, Abstände manuell zu steuern. Beispielsweise sorgt ''\verb|\\|'' für einen Zeilenumbruch, alternativ können Sie auch \verb|\newline| verwenden.\\
Für einen Seitenumbruch wird \verb|\newpage| oder auch \verb|\clearpage| verwendet.

\vspace{3mm}
\noindent {\bf Horizontale Abstände} können mit \verb|\hspace{...}| gesteuert werden, wobei Sie bei \verb|{...}| die Längenangabe inklusive Einheit definieren müssen. Soll der ganze Rest der Zeile leer gelassen werden, so können sie auch mit \verb|\hfill| arbeiten.

\noindent {\bf Vertikale Abstände} (zwischen Zeilen oder Absätzen) können mit \verb|\vspace{...}| gesteuert werden, wobei Sie bei \verb|{...}| die Längenangabe inklusive Einheit definieren müssen. Soll etwas zuunterst auf einer Seite geschrieben werden, so können Sie davor mit \verb|\vfill| arbeiten.

\vspace{3mm}
\noindent Wenn Sie vermeiden möchten, dass eine Zeile automatisch eingerückt wird, so können Sie \verb|\noindent| verwenden.

\newpage
\subsection{Textformatierungen}
Der Code \verb|\emph{Dies ist kursiv.}| erzeugt folgende Ausgabe: \emph{Dies ist kursiv}.

\vspace{2mm}
\noindent Andere häufige Anweisungen zur Textgestaltung sind:

\vspace{4mm}
\begin{tabular}{ll}
\verb|\textbf{argument}| & \textbf{Fetter Text} \\
\verb|\textit{argument}| &  \textit{kursiver Text} \\
\verb|\underline{argument}| &  \underline{unterstrichener Text} \\
\verb|{\tiny argument}| & \tiny{winziger Text} \\
\verb|{\small argument}| & {\small kleiner Text} \\ 
\verb|{\large argument}| & {\large grosser Text} \\
\verb|{\huge argument}| & {\huge riesiger Text}
\end{tabular}

\vspace{4mm}
\noindent Es ist auch möglich, Texte farbig zu gestalten. Dazu müssen Sie zunächst folgendes Package aufführen: \verb|\usepackage{xcolor}|. Dies können Sie beispielsweise auf Zeile 3 schreiben.\\
 Danach erzeugt der Code ''\verb|\textcolor{blue}{farbiger Text}|''  folgende Ausgabe: \textcolor{blue}{farbiger Text}

\vspace{4mm}
\begin{ex} \label{ex:1}
Ändern Sie zuerst in Overleaf den Abschnittstitel \emph{''Introduction''} zu \emph{''Aufgabe \ref{ex:1}''}.\\
Versuchen Sie im Anschluss den unten abgedruckten Text zu erzeugen:
\end{ex}

\vspace{2mm}
\noindent \textbf{Das ist ein formatierter Text.} \emph{Das ist ein formatierter Text.} \underline{Das ist ein formatierter Text.}\\
\textcolor{red}{Das ist ein formatierter Text.} \textcolor{green}{Das ist ein formatierter Text.} \textcolor{blue}{Das ist ein formatierter Text.}\\
\noindent {\tiny Das ist ein formatierter Text. Das ist ein formatierter Text. Das ist ein formatierter Text.}

\vspace{1cm}
\noindent {\huge Das ist ein formatierter Text. Das ist ein formatierter Text. Das ist ein formatierter Text.}