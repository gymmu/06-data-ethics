\section{Listen}
Es gibt drei Hauptarten von Listen, welche Sie nun kennenlernen werden: nummerierte Listen, Aufzählungslisten und beschreibende Listen.

\subsection{Nummerierte Listen}
Sie haben bereits weiter oben Nummerierte Listen gesehen. Die Aufgaben, welche Sie gelöst haben, wurden dadurch strukturiert und in kleinere Teile zerlegt.
Eine nummerierte Liste kann mit dem Befehl \verb|\begin{enumerate}| folgendermassen erzeugt werden:
\begin{verbatim}
\begin{enumerate}
\item Beginnen Sie jede nummerierte Liste mit einem \begin{enumerate}-Befehl.
\item Fügen Sie Ihre Listenelemente mit dem \item-Befehl hinzu.
\item Vergessen Sie nicht, Ihre Liste zu beenden.
\end{enumerate}
\end{verbatim}
Obiger Code wird im PDF-Dokument folgendermassen dargestellt:
\begin{enumerate}
\item Beginnen Sie jede nummerierte Liste mit einem \verb|\begin{enumerate}|-Befehl.
\item Fügen Sie Ihre Listenelemente mit dem \verb|\item|- Befehl hinzu.
\item Vergessen Sie nicht, Ihre Liste zu beenden.
\end{enumerate}


\subsection{Aufzählungslisten}
Manchmal möchten Sie jedoch eine Aufzählungsliste, auch bekannt als unnummerierte Liste. Aufzählungslisten funktionieren ähnlich wie nummerierte Listen.

\begin{verbatim}
\begin{itemize}
\item Beginnen Sie jede Aufzählungsliste mit einem \begin{itemize}-Befehl.
\item Fügen Sie Ihre Listenelemente mit dem \item-Befehl hinzu.
\item Vergessen Sie nicht, Ihre Liste zu beenden.
\end{itemize}
\end{verbatim}
Obiger Code wird im PDF-Dokument folgendermassen dargestellt:
\begin{itemize}
\item Beginnen Sie jede Aufzählungsliste mit einem \verb|\begin{itemize}|-Befehl.
\item Fügen Sie Ihre Listenelemente mit dem \verb|\item|-Befehl hinzu.
\item Vergessen Sie nicht, Ihre Liste zu beenden.
\end{itemize}


\subsection{Beschreibungslisten}
Eine weitere Möglichkeit bietet die Beschreibungsliste:
\begin{verbatim}
\begin{description}
\item[Aufzählungslisten] Nützlich für Listen von Materialien
\item[Nummerierte Listen] Nützlich für Verfahren
\item[Beschreibungslisten] Nützlich für Vokabeln
\end{description}
\end{verbatim}
Obiger Code wird im PDF-Dokument folgendermassen dargestellt:

\begin{description}
\item[Aufzählungslisten] Nützlich für Listen von Materialien
\item[Nummerierte Listen] Nützlich für Verfahren
\item[Beschreibungslisten] Nützlich für Vokabeln
\end{description}

\vspace{5mm}
\noindent Das spezielle an Beschreibungslisten ist, dass Sie ein Aufzählungssymbol Ihrer Wahl verwenden können:
\begin{verbatim}
\begin{description}
\item [$\star$] Ein Sternsymbol als Aufzählungssymbol.
\item [\#]  Ein Hashtag als Aufzählungssymbol.
\item [$\Rightarrow$] Ein Pfeil als Aufzählungssymbol.
\end{description}
\end{verbatim}
Obiger Code wird im PDF-Dokument folgendermassen dargestellt:
\begin{description}
\item [$\star$] Ein Sternsymbol als Aufzählungssymbol.
\item [\#]  Ein Hashtag als Aufzählungssymbol.
\item [$\Rightarrow$] Ein Pfeil als Aufzählungssymbol.
\end{description}

\vspace{3mm}
\begin{ex} \label{ex:4}
Fügen Sie wieder einen neuen Abschnittstitel \emph{''Aufgabe \ref{ex:4}''}  in Overleaf hinzu.\\
Erstellen Sie eine Liste mit vier Einträgen, welche als Aufzählungssymbol einen Kreis ($\circ$) verwendet. Tipp: \LaTeX\ -Cheat-Sheet
\end{ex}